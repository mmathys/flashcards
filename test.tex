\documentclass[]{article}
\usepackage{pgfpages}
\setlength{\parindent}{0pt}%Wenn Absatzabstand, dann Einzug unnötig

\usepackage[paperwidth=.5\paperwidth,paperheight=.25\paperheight]{geometry}
\pagestyle{empty}
\thispagestyle{empty}
\pgfpagesuselayout{8 on 1}[a4paper]
\makeatletter
\@tempcnta=1\relax
\loop\ifnum\@tempcnta<9\relax
\pgf@pset{\the\@tempcnta}{bordercode}{\pgfusepath{stroke}}
\advance\@tempcnta by 1\relax
\repeat
\makeatother

\newenvironment{flashcard}[2][]{%
\noindent  \textsc{#1}

\vfill
\centerline{{\Large\emph{#2}}}
\vfill
\newpage
}
{\newpage}

\newcommand{\definition}[1]{%
	\begin{flashcard}[Definition]{#1}
		
	\end{flashcard}
	}
	
\newcommand{\term}[1]{%
	\noindent  \textsc{Term}
	
	\vfill
	\vspace*{\stretch{1}}
	#1
	\vspace*{\stretch{1}}
	

	\vfill
	\newpage
}

\usepackage[utf8]{inputenc}

\usepackage{amsfonts}
\usepackage{amsmath}

\begin{document}

\definition{Hornhaut fkt}
\definition{Vordere Augenkammer mit Wasser}
\definition{Linse}
\definition{Ciliarmuskel}
\definition{Regenbogenhaut=Iris}
\definition{Glaskörper}
\definition{Netzhaut}
\definition{Pigmentschicht}
\term{Zusammen mit Hornhaut als Sammelllinsee}
\term{Durch Wölbung wirkt als Sammellinse}
\term{Stellen die Linse ein}
\term{Bricht das Licht so, dass ein scharfes Bild entsteht}
\term{Formerhaltung des Auges}
\term{Lässt mehr oder weniger Licht durch}
\term{Schutz gegen Reflexionen im Auge}
\term{Hell-Dunkel mit Stäbchen und Farbensehen mit Zäpfchen}
\definition{Aderhaut}
\definition{Weisse Aderhaut}
\definition{Blinder Fleck}
\definition{Sehnerv}
\definition{Gelber Fleck }
\definition{Augenbindehaut}
\definition{Augenmuskeln}
\definition{Fettpolster}
\term{Gibt dem Auge die Form}
\term{Ernährt Auge}
\term{Weiterleitung ans Gehirn}
\term{Keine Nerven, Dort gehen Nervenfasern raus}
\term{Verbindet Augenapfel mit Gesichtshaut}
\term{Ort mit am Meisten Sehnerven, schärfste Abbildung}
\term{Weiche Lage des Auges}
\term{Stelllen die Auge ein}
\definition{Tränendrüsen}
\definition{Stäbchen Farbstoff}
\definition{Zäpfchen Farbstoffe}
\definition{Fototransduktion}
\definition{Adaption des Augen}
\definition{Myopie}
\definition{Hyperopie}
\definition{Akkommodation Auge}
\term{Sehpurpur = Rhodopsin}
\term{Bewässerung des Auges}
\term{Umsetzung Lichtreize in bioelektrische Signale: Guanosinmonosphat aktiviert und Hyperoplarisation}
\term{cis-Retinal (für wenig Licht), trans-Retinal (für viel Licht)}
\term{Kurzsichtigkeit; Auge ist relative zur Brechkraft zu lang (kurzsicht --> lang)}
\term{Irisblende}
\term{Anpassung an Weite. Linse durch Ciliarmuskel verändert}
\term{Weitsichtigkeit, Auge ist relativ zur Brechkraft zu kurz (langsicht --> kurz)}
\definition{Alterssichtigkeit}
\definition{}
\definition{}
\definition{}
\definition{}
\definition{}
\definition{}
\definition{}
\term{}
\term{Elastizität der Linse nicht mehr gut}
\term{}
\term{}
\term{}
\term{}
\term{}
\term{}




\end{document}