\documentclass[]{article}
\usepackage{pgfpages}
\setlength{\parindent}{0pt}%Wenn Absatzabstand, dann Einzug unnötig

\usepackage[paperwidth=.5\paperwidth,paperheight=.25\paperheight]{geometry}
\pagestyle{empty}
\thispagestyle{empty}
\pgfpagesuselayout{8 on 1}[a4paper]
\makeatletter
\@tempcnta=1\relax
\loop\ifnum\@tempcnta<9\relax
\pgf@pset{\the\@tempcnta}{bordercode}{\pgfusepath{stroke}}
\advance\@tempcnta by 1\relax
\repeat
\makeatother

\newenvironment{flashcard}[2][]{%
\noindent  \textsc{#1}

\vfill
\centerline{{\Large\emph{#2}}}
\vfill
\newpage
}
{\newpage}

\newcommand{\definition}[1]{%
	\begin{flashcard}[Definition]{#1}
		
	\end{flashcard}
	}
	
\newcommand{\term}[1]{%
	\noindent  \textsc{Term}
	
	\vfill
	\vspace*{\stretch{1}}
	#1
	\vspace*{\stretch{1}}
	

	\vfill
	\newpage
}

\usepackage[utf8]{inputenc}

\usepackage{amsfonts}
\usepackage{amsmath}

\begin{document}

\definition{Thema 01}
\definition{sekundärer aktiver Transport}
\definition{geschlossener Blutkreislauf}
\definition{N Kreislauf}
\definition{Chromosom}
\definition{Agrobacterium Übertragungsmethode}
\definition{N. Invasive Methoden}
\definition{rechte Hemisphäre Fkt}
\term{Zurückströmen der Ionen das zu transportierende Substrat eingesetzt.}
\term{NO VOCI THANK GOD}
\term{N --> NH4 --> NO2- --> NO3- --> N}
\term{Das Blut fliesst in den Röhren}
\term{Durch Zellverletzung schleust Agrobacterium Stoffe ein.}
\term{2 Chromatide}
\term{Vergleichen, Zusammenfassen, Betrachten}
\term{Ultraschall}
\definition{Zellwand Material}
\definition{Endocytose}
\definition{Blut Kreislauf}
\definition{Ammonifikation}
\definition{Chromatid}
\definition{Vektor (-DNA)}
\definition{Invasive Methoden}
\definition{Lateralisierung}
\term{Aufnahme von Stoffen über Endosome. Zellmembran stülpt sich an einer Stelle so weit ein, bis der Stoff umschlossen ist. Zb Proteine und Polysaccharide in Zelle transportiert.}
\term{Polysaccharide}
\term{N --> NH4 Bakterien bauen organische Stoffe Ammoniumionen um.}
\term{Herz --> Arterien --> Kapillaren --> Venen --> Herz}
\term{Plasmide, Transport}
\term{1 Strang.}
\term{Nicht-Vernetzt-Sein der Hemisphären"Thema 13"}
\term{Fruchtwasseruntersuchung}
\definition{Cytoplasma}
\definition{Phagocytose}
\definition{einfacher Blutkreislauf}
\definition{Nitrifikation}
\definition{homologe Chromosomen}
\definition{Plasmide fkt}
\definition{Ultraschall}
\definition{Hornhaut fkt}
\term{Endocytose von Festteilchen}
\term{Durch Zellmembran begrenzt}
\term{NH4 --> NO2- / NO2- ---> NO3- Bakterien gewinnen Energie aus Ammoniumionen, es braucht aber O2}
\term{Zb bei Fischen; der Lungenkreislauf ist vom Körperkreislauf NICHT getrennt.}
\term{Ringförmige DNA-Moleküle, auf für Abwehr}
\term{Chromosomen gleicher Grösse und Gestalt, oft stammt eines vom Vater und eines von der Mutter.}
\term{Durch Wölbung wirkt als Sammellinse}
\term{Fruchtwassermenge}
\definition{Bakterien Zellaufbau}
\definition{Pinocytose}
\definition{doppelter Blutkreislauf}
\definition{Stickstoff-Assimiliation}
\definition{Allel}
\definition{Selektion transgener Zellen wieso}
\definition{Serumuntersuchung (Alpha-Feto-Protein)}
\definition{Vordere Augenkammer mit Wasser}
\term{Endocytose von Flüssigteilchen}
\term{Keinen Zellkern, sondern nur Kernäquivalent --> Nucleoid}
\term{NO3- --> N}
\term{der Lungenkreislauf ist vom Körperkreislauf getrennt.}
\term{Weil Erfolgsquote des Einschleusen nicht sehr gut}
\term{Ausprägung eines Gens, das sich auch an einem anderen Ort befindet.}
\term{Zusammen mit Hornhaut als Sammelllinsee}
\term{Bei Missbildungen hat es mehr AFP. Tripletest: Hormon AFP}
\definition{Nucleoid}
\definition{Exocytose}
\definition{Sinusknoten}
\definition{Nitratammonifikation}
\definition{homozygot}
\definition{Antibiotika-Resistenz-Test}
\definition{Praena-Test}
\definition{Linse}
\term{Abgabe von Stoffen}
\term{Kernäquivalent}
\term{NO3- --> NH4 Aus irgendeinem Grund von Phytoplankton gemacht}
\term{in Verbindung mit vegetativen Nervensystem, erregt Herz, }
\term{Vektor mit Amphicilin und Tetracilin-Resistenzgen; in Vektor Fremd-DNA eingebaut (Fremd-->Tetracilin-Teil rausgeschnitten); Amphiclin; dann Amphicilien+Tetracylin Lösung für Aussortierung}
\term{reinerbig, Allele tragen gleiche Information}
\term{Bricht das Licht so, dass ein scharfes Bild entsteht}
\term{Aus dem Blut fetale DNA herausgefiltert und auf Trisonomie geüorüft}
\definition{Prokaryoten}
\definition{primäres Lyosom}
\definition{Systole1}
\definition{oligotroph, eutroph}
\definition{heterozygot}
\definition{Biotechnik}
\definition{Fruchtwasseruntersuchung}
\definition{Ciliarmuskel}
\term{Aus Golgi-Vesikeln gebildet und enthält Verdauungsenzyme, fusioniert (Membranfluss) mit sekundären Lyosom}
\term{Zb Bakterien, haben keinen Zellkern, sondern nur Nucleoid.}
\term{mineralstoff armer, reicher See.}
\term{Ziehen sich Hauptkammern zusammen, Vorkammer öffnen sich}
\term{technische Nutzung biologischer Fähigkeiten}
\term{mischerbig, Allele tragen NICHT NCIHT die gleiche Information}
\term{Stellen die Linse ein}
\term{Mit Nadel Fruchtwasser entnehmen}
\definition{Plasmide}
\definition{sekundäres Lyosom}
\definition{Diastole}
\definition{oligotrophe Seen}
\definition{hemizygot.}
\definition{Gentechnik}
\definition{Chorionzottenbiopsie}
\definition{Regenbogenhaut=Iris}
\term{Partikel werden verdaut und ausgeschieden durch Exocytose}
\term{kleine Ringe der Kernsäure DNA}
\term{Wenig Phytoplankton wenig Mineralien, mehr Destruenten; Zehrschicht Sauerstoffreich}
\term{Ziehen sich Vorkommern zusammen}
\term{Veränderung Erbinformation}
\term{Hm. Das Allel tritt nur einmal auf.}
\term{Lässt mehr oder weniger Licht durch}
\term{Aus Placenta Gewebe entnehmen}
\definition{Zwei Gruppen Prokaryoten}
\definition{rezeptorenvermittelte Endocytose}
\definition{Systole2}
\definition{eutrophe Seen}
\definition{Gonosome}
\definition{Reproduktionstechnik}
\definition{PID}
\definition{Glaskörper}
\term{selektive Aufnahme von Makromolekülen mittels Rezeptorproteine,"Thema 03}
\term{Bakterien und Archaeen}
\term{Viel Phytoplanktion viel Mineralien, mehr Phytoplanktion Nährschicht Sauerstoffreich}
\term{Mit Systolendruck Blut senden}
\term{Manipulation von Reproduktionsvorgängen}
\term{Die geschlechtsspezifischen Chromosomen: X und Y. Frau: XX, Mann: XY}
\term{Formerhaltung des Auges}
\term{Präimplantationsdiagnostik}
\definition{Eukaryoten}
\definition{DNA <--> RNA}
\definition{Lymphsystem}
\definition{Bruttoprimärproduzenten}
\definition{Autosomen}
\definition{Insulin Wie hergestellt}
\definition{PID Vorgehen}
\definition{Netzhaut}
\term{C2-Molekül DNA hat H anstelle OH}
\term{Besitzen echten Zellkern}
\term{Verwenden 1\% der Sonnenenergie für Biomasseproduktion}
	\term{Lymphe = Blut ohne Blutzellen}
	\term{Isolierung des Insulins von menschlicher Bauchspeicheldrüse, Vektor ist Bakterienplasmid, Kombination durch RE und Ligase, Selektion, Vermehrung, Isolation, Reinigung}
	\term{Die geschlechtsunspezifischen Chromosomen}
	\term{Hell-Dunkel mit Stäbchen und Farbensehen mit Zäpfchen}
	\term{Aus Eierstock Eizelle entnehmen}
	\definition{Ribosom}
	\definition{Grundeinheit}
	\definition{Entstehung Blutplasma}
	\definition{Kosument Aufnahme}
	\definition{Erbgang von Allelen (4)}
	\definition{Gel-Elektrophorese}
	\definition{PID Pro/Kontra}
	\definition{Pigmentschicht}
	\term{Nucleotid}
	\term{makromolekulare Komplexe aus Proteinen und RNA}
	\term{Nur 10\% der Energie wird aufgenommen}
		\term{Durch hohen Druck in arteriellen Kapillaren wird das Blut GEFILTERT}
		\term{Auftrennung von Bruchstücken nach Länge in elektrischen Feld, --> Banden (Färbung)}
		\term{dominant, rezessiv, intermediär, kodominant}
		\term{Schutz gegen Reflexionen im Auge}
		\term{Krankheiten vermeiden; Designer-Babies: Das Kind so designen}
		\definition{Kompartimente}
		\definition{Zucker der RNA}
		\definition{Reabsorption}
		\definition{Nahrungspyramide}
		\definition{dominant}
		\definition{DNA-Sequenzanalyse}
		\definition{Thema 12}
		\definition{Aderhaut}
		\term{Ribose}
		\term{membranbegrenzte Räume in der Zelle, gebraucht um abzugrenzen}
		\term{Darstellen, wieviel Energie weitergegeben wird. Zb 100\% 10\% 1\% 0.1\%}
			\term{Blutplasma gelangt durch Osmose zurück in Blut}
			\term{DNA Stränge denaturalisieren, 4 Reagenzgläser, je ein Typ ddNTP, bei Synthese Abbruch bei Einbau von ddNTP, Aufteilung nach Gel-Elektrophorese}
			\term{überdeckend}
			\term{Ernährt Auge}
			\term{NO VOCI THANK GOD}
			\definition{Mitochondrien}
			\definition{Zucker der DNA}
			\definition{Lymphgefässsystem Funktion}
			\definition{r-Selektiert}
			\definition{rezessiv}
			\definition{Fluoreszenzsequenzierung}
			\definition{Rezeptorzellen}
			\definition{Weisse Aderhaut}
			\term{Desoxyribose}
			\term{Gebilde in Zelle, die in Eukaryoten vorkommt und Erbsubstanz enthält. Auf Oberfläche von Matrix bildung von ATP; Krafwerke von Zellen}
			\term{Lebensraum schwach besiedelt, Selektion begünstig Lebewesen mit schneller Vermehrungsrate}
			\term{Abtransport von Gewebesflüssigkeit}
			\term{Für jede Base eigener Flureszenzfarbstoff; automatische Fluoreszenzierung}
			\term{übergedeckt}
			\term{Gibt dem Auge die Form}
			\term{Sinneszellen, die auf ädequate Reizart ansprechen. Wandel Informationen aus Umwelt in Signale um}
			\definition{ATP}
			\definition{DNA Basen}
			\definition{Lymphknoten Funktion}
			\definition{K-Selektiert}
			\definition{intermediär}
			\definition{genetischer Fingerabdruck}
			\definition{Zwei Arten Nervenzellen}
			\definition{Blinder Fleck}
			\term{A, G / C, T; Adenosin, Guanin / Cytosin, Thymin}
			\term{Adenosintriphospat}
			\term{Lebensraum an Grenze seiner Kapazität, Selektion begünstigt Lebewesen mit guten Anfangsbedingungen.}
			\term{Reinigungs- und Abwehrarbeit}
			\term{Merkmale der DNA, die für ein Individuum charakteristisch sind}
			\term{Allele nehmen Mischform an.}
			\term{Keine Nerven, Dort gehen Nervenfasern raus}
			\term{afferent, efferent}
			\definition{Endoplasmatische Reticulum}
			\definition{RNA hat nicht Thymin. Was hat es anstelle?}
			\definition{Ödem}
			\definition{Kennzeichen r-/K-Selektiert}
			\definition{kodominant}
			\definition{SNP}
			\definition{afferent}
			\definition{Sehnerv}
			\term{U Uracil}
			\term{Verzweigtes Kanalsystem, Geht direkt in die Zellhülle des Reticulums über, raues ER, glattes ER}
			\term{Zeit bis Geschlechtsreife, Anzahl Nachkommen, Dicke Eierschalen, Lebensdauer, Sterblichkeit}
			\term{Der Abtransport on Gewebesflüssigkeiten funktioniert nicht mehr richtig, es kommt zu einem Stau"Thema 05}
			\term{Punktmutationen in DNAs}
			\term{Allele manifestieren sich nebeneinander.}
			\term{Weiterleitung ans Gehirn}
			\term{sensorische Nervenzellen: Leiten Signale an Gehirn weiter}
			\definition{raues ER <--> glattes ER}
			\definition{Doppelhelix-Modell}
			\definition{äussere Atmung}
			\definition{Bruttoprimärproduktion}
			\definition{gonosomal, autosomal}
			\definition{RFLP}
			\definition{efferent}
			\definition{Gelber Fleck }
			\term{DNa liegt als Doppelstrange auf zwei langen Fäden vor, die geschraubt sind; Basen sind durch H-Bindungen gebunden}
			\term{Auf Membranoberfläche Ribosomen <--> Keine Ribosomen}
			\term{Menge an Kohlenstoff, welche Pflanzen bei Fotosynthese bilden.}
			\term{Austausch der Atemgase des Körpers mit der Umgebung}
			\term{Restriktionsfragment-Längenpolymorphismen, SNPs, die innerhalb der Erkennungssequenz liegen}
			\term{Allel liegt auf Gonosom, Autosom}
			\term{Ort mit am Meisten Sehnerven, schärfste Abbildung}
			\term{motorische Nervenzellen: Leiten Signale von Gehirn an Muskeln.}
			\definition{Ribosome}
			\definition{Basen paaren sich wie:}
			\definition{Hautatmung}
			\definition{Nettoprimärproduktion}
			\definition{P, F1, F2}
			\definition{VNTRs}
			\definition{Gliazellen, Fkt}
			\definition{Augenbindehaut}
			\term{AT und GC}
			\term{kugelige Partikel, aus RNA und Proteinen}
			\term{Pflanzen Energie abzüglich der selbst verbrauchten, = was sie hinterlassen an Biomasse}
			\term{Sauerstoff aufnehmen durch Oberfläche der Haut}
			\term{Variable Number Tandem Repeats: Wiederholte DNA-Sequenzen}
			\term{Parental-, 1-Filial-, 2-Filialgeneration}
			\term{Verbindet Augenapfel mit Gesichtshaut}
			\term{Bindegewebszellen; Stütz- und Hüllfunktion; Stoffwechsel der Nerven}
			\definition{Zellwand: Aufbau  Funktion}
			\definition{Zwei Enden}
			\definition{Atmung Fisch}
			\definition{Effekt verschmutztes Wasser}
			\definition{Trisonomie 21}
			\definition{STR}
			\definition{Nervensystem Def}
			\definition{Augenmuskeln}
			\term{C5 und C3: Phospatrest am C5-Atom; entgegengesetzte Richtung C3-Atom}
			\term{Material: Mikrofibrillen, aus Cellulose --> Mikrofilienbündel.}
			\term{Bakterien vermehren sich stark, mehr NH4, NO2, NO3, Phosphat-Ionen freigesetzt; Schwebstoffe verhindern Fotosynthese, Sauerstoff sinkt, keine Fische}
			\term{Kiemen}
			\term{Short Tandem Repeats: Kurze VNTR.}
			\term{Zusätzliches Chromosom 21}
			\term{Stelllen die Auge ein}
			\term{Nervenzellen + Gliazellen}
			\definition{Zellverbindungen}
			\definition{Richtung}
			\definition{Atmung Insekten}
			\definition{anaerobe Verhältnhisse}
			\definition{Nondisjunction}
			\definition{Genchip}
			\definition{Soma Fkt}
			\definition{Fettpolster}
			\term{Immer 5' --> 3'; komplementärer Strang 3' --> 5'}
			\term{Plasmodesmen bei Pflanzen, gap junctions bei Tieren}
			\term{O2-Konzentration verringert sich}
			\term{Tracheen}
			\term{Zur Analyse von DNA}
			\term{Fehlendes auseinanderweichen von Chromosomen}
			\term{Weiche Lage des Auges}
			\term{Zellkörper der Nervenzelle}
			\definition{Plasmodesmen}
			\definition{Antiparallellität der Stränge}
			\definition{Atmung Wirbeltiere}
			\definition{Selbstreinigungskraft des Gewässers}
			\definition{TurnerSyndrom}
			\definition{Transgene Pflanze Bsp}
			\definition{Dendriten Ftk}
			\definition{Tränendrüsen}
			\term{5' 3' vs 3' 5' Raumstruktur / Anordnung ist so möglich}
			\term{zyliderförmige Kanäle in pflanzlichen Zellen}
			\term{Fast ausschliesslich Bakterien}
			\term{Lunge mit Bronchien}
			\term{Transgene Sojabohnen}
			\term{X0}
			\term{Bewässerung des Auges}
			\term{Aufnahme von Signalen}
			\definition{gap junctions}
			\definition{Gen}
			\definition{Vitalkapazität}
			\definition{Indikatorlebewesen}
			\definition{Poly-X-Frauen}
			\definition{Pro Sojabohnen}
			\definition{Axon=Neurit Fkt}
			\definition{Stäbchen Farbstoff}
			\term{Erbanlage, bestimmte Fähigkeiten}
			\term{tierische Zellen, Kommunikations-Kontaktstellen}
			\term{Lebewesen, die unter besonderen chemischen Bedinungen leben}
			\term{Atemzugvolumen, 4-5l}
			\term{proteinreich und billig.}
			\term{XXX, XXXX}
			\term{Sehpurpur = Rhodopsin}
			\term{Länge der Nervenzelle}
			\definition{Desmosome}
			\definition{Chromosom}
			\definition{Wieso Atmen in grosser Höhe schwierig}
			\definition{Saprobien}
			\definition{Klinefelter-Männer}
			\definition{Kontro Sojabohnen}
			\definition{Schwannschen Zelle}
			\definition{Zäpfchen Farbstoffe}
			\term{undefined}
			\term{Haftpunkte bei Zellenverbindungen}
			\term{Lebewesen, die durch physiologische Anpassung unter verfaulten Zuständen leben können, Saprobienindex}
			\term{Weil Partialdruck des Sauerstoffes klein ist}
			\term{Wälder abgeholzt; Grosse Unternehmen dominieren, vertreiben Bauern; Monsanto: Abhängig von Pestiziden; }
			\term{XXY, XXXY}
			\term{cis-Retinal (für wenig Licht), trans-Retinal (für viel Licht)}
			\term{Gliazelle, die eine Nervenzelle umgeben}
			\definition{tight junctions}
			\definition{Karyogramm}
			\definition{Wieso Atmen in grosser Tiefe schwierig}
			\definition{Trophiestufe}
			\definition{Diplo-Y-Männer (diplo = doppelt)}
			\definition{Transgenes Tier Bsp}
			\definition{Synapsen Fkt}
			\definition{Fototransduktion}
			\term{Ordnen der Chromosomen nach Grösse und Aussehen}
			\term{Zellen werden zusammen gehalten, gleichzeitig abgedichtet.}
			\term{Mineralstoffgehalt}
			\term{Weil Atemhöhle kollabieren}
			\term{Gen-Lachs: Aus Paarung zweier Fische gemacht.}
			\term{XYY, XXYY, XXXYY}
			\term{Umsetzung Lichtreize in bioelektrische Signale: Guanosinmonosphat aktiviert und Hyperoplarisation}
			\term{Endknöpfchen, docken bei weiteren Nervenzellen oder Muskeln an.}
			\definition{Unterschied Desmosom tight junctions}
			\definition{Allele}
			\definition{Wieso Atmen in Trockenheit schwierig}
			\definition{Reinigung Abwasser: 3 Stufen}
			\definition{Replikation DNA Enzyme}
			\definition{Gen-Lachs Pro}
			\definition{Myelin}
			\definition{Adaption des Augen}
			\term{Jedes Gen liegt auf zwei Allelen eines Chromosoms.}
			\term{Desmosom ist nicht so dicht, hat noch ein wenig space.}
			\term{mechanisch, biologisch, chemisch}
			\term{Für Kiemenatmer, weil Kiemen zusammenfallen"Thema 06}
			\term{Wächst schneller}
			\term{Helicase, Primase, (DNA Polymerase III | DNA Polyermase III, DNA-Polymerase I, DNA-Ligase)}
			\term{Irisblende}
			\term{Hülle um Nervenzelle}
			\definition{Cytoskelett}
			\definition{Diploid}
			\definition{abiotische Umweltfaktoren}
			\definition{Mechanische Reinungsstufe}
			\definition{Helicase Fkt}
			\definition{Gen-Lachs Kontra}
			\definition{Markscheide}
			\definition{Myopie}
			\term{Zustand eines Chromosomensatzes: Er enthält einen Chromosomensatz von Vater und Mutter}
			\term{Cytoplasma ist von Proteinfilamenten durchzogen. mechanische festigkeit, plasmaströmungen, muskelkontraktionen}
			\term{Rechenwerk, Steine am Boden ablagern, Fett an der Oberfläche wegnehmen}
			\term{Temp, Licht, Feuchtigkeit}
			\term{ethische Gründe, schmerzen für Tiere; gv-Tiere mit wilden Tieren kreuzen; "Thema 10}
			\term{Aufspalten der Einzelstränge}
			\term{Kurzsichtigkeit; Auge ist relative zur Brechkraft zu lang (kurzsicht --> lang)}
			\term{Myelinscheide, um Nervenzellen}
			\definition{Mikrotubuli}
			\definition{Genom, Genotyp}
			\definition{Biotop}
			\definition{Biologische Reinigungsstufe}
			\definition{Primase Fkt}
			\definition{homozygot}
			\definition{Ranviersche Schnürringe}
			\definition{Hyperopie}
			\term{Gesamtheit der Gene eines Zellkern}
			\term{röhrenförmige Proteinfilamente}
			\term{Belebtschlammbecken: aerob: Nitrifikation NH4 --> NO2- --> NO3-; anaerob: Denitrifikation NO3- --> NO2- --> NH4 --> N2;}
			\term{Lebensraumens einer Art}
			\term{reinerbig}
			\term{Ansetzen der RNA-Primer}
			\term{Weitsichtigkeit, Auge ist relativ zur Brechkraft zu kurz (langsicht --> kurz)}
			\term{Platz zwischen Markscheide und Axon}
			\definition{Centriolen}
			\definition{Phänotyp}
			\definition{biotische Umweltfaktoren}
			\definition{Chemische Reinigungsstufe}
			\definition{Enzyme 5' --> 3' vs 3' --> 5 Gabel}
			\definition{heterozygot}
			\definition{Strickleiternervensystem}
			\definition{Akkommodation Auge}
			\term{Das Erscheinungsbild der Gene: Zusätzlich zum Genotyp wirken auch noch weitere Faktoren mit}
			\term{Bestehen aus Mikrotubuli, bei Zellteilung eine Rolle}
			\term{Durch hinzugabe von Eisen- und Aluminiumsalze werden die Phospationen ausgefällt und lagern sich ab am Boden}
			\term{Einwirkung, denen ein Lebewesen durch andere Lebewesen ausgesetzt ist}
			\term{mischerbig}
			\term{5' --> 3' : DNA-Polyermase III. 3' --> 5': DNA-Polymerase III, DNA-Polyermase I, DNA-Ligase.}
			\term{Anpassung an Weite. Linse durch Ciliarmuskel verändert}
			\term{Zwei Hauptnervenstränge + Querverbände, zb Plattwürmer}
			\definition{Zellkern}
			\definition{Mitose}
			\definition{Biozönose}
			\definition{Schlamm}
			\definition{DNA-Polyermase III Fkt}
			\definition{1. Mendelsche Regel}
			\definition{Zentralnervensystem}
			\definition{Alterssichtigkeit}
			\term{Zellkernteilung}
			\term{Nucleus, RNA, DNA}
			\term{Kann verbrannt werden, Gas kann verbrannt werden"Thema 07}
			\term{Lebensgemeinschaft aller Lebewesen eines Biotops}
			\term{Uniformitätsregel: 2 Individuen einer Art kreuzen, die eine Eigenschaft unterscheiden, für das sie reinerbig sind, sind die Nachkommen identisch in dieser Eigenschaft}
			\term{Synthese neuer komplementärer DNA}
			\term{Elastizität der Linse nicht mehr gut"Thema 14"}
			\term{Ganglien bilden räumlich und funktional eine Einheit}
			\definition{Karyoplasma, Cytoplasma}
			\definition{Stadien Mitose}
			\definition{Ökosystem}
			\definition{Fotosynthese Gleichung}
			\definition{Okazaki-Fragment}
			\definition{2. Mendelsche Regel}
			\definition{Ganglien}
			\definition{Proximate Ursache}
			\term{PMATI: Prophase, Metaphase, Anaphase, Telophase, Interphase}
			\term{Inneres, Äusserer Inhalt}
			\term{6 CO2 + H2O -> C6H12O6 + 6 O2}
			\term{Biozönose+Biotop}
			\term{Spaltungsregel: Kreuzt man die Invididuen 1. Filialgeneration, so teilt sich der Phänotyp im dominant-rezessiven Erbgang in 25\% 75\%}
			\term{Auf 3'-->5' Seite keine kontinuierliche Synthase möglich, darum DNA-Polymerase III inkontinuierliche Synthase. --> Okazaki-Fragment entstehen}
			\term{Ursachen, die beschreiben, wodurch ein Verhalten vorgerufen wird (Hormone, Körper)}
			\term{Ansammlung von Nervenzellen}
			\definition{Vakuole}
			\definition{Prophase}
			\definition{Biosphäre}
			\definition{Biomasse}
			\definition{DNA-Polyermase I Fkt}
			\definition{3. Mendelsche Regel}
			\definition{Mensch ZNS}
			\definition{Ultimate Ursachen}
			\term{2-Chromatid-Chromosomen verknäuelt --> verkürzen sich (Kondensation) }
			\term{Mit Flüssigkeit gefüllte Hohlräume in pflanzlichen Zellen}
			\term{Zucker, Holz, Fette, Eiweisse}
			\term{Alle Ökosysteme}
			\term{Unabhängigkeitsregel: Die verschiedenen Merkmale sind frei miteinander kombinierbar}
			\term{Ersetzen der RNA-Primer durch DNA-Komplementärbasen}
			\term{Biologische Bedeutung, was nützt sie? Wie ist sie evolutiv entstanden?}
			\term{Gehirn + Rückenmark}
			\definition{Turgor}
			\definition{Metaphase}
			\definition{Optimum}
			\definition{sichtbares Licht}
			\definition{DNA-Ligase}
			\definition{X-Chromosomengebundene Kreuzung}
			\definition{Mensch peripheres NS}
			\definition{Etho-Phylogenie}
			\term{2-Chromatid-Chromosomen ordnen sich an der Äquatorialplatte}
			\term{Innendruck der Zelle, durch Vakuole}
			\term{400-700nm}
			\term{Temp. mit Höchster Aktivität}
			\term{Möglichkeit 1: dominante Gene liegen auf X X fem. Möglichkeit 2: dominante Gene liegen auf X male}
			\term{Verbindet Okazaki-Fragmente}
			\term{Angeboren oder Erlernt?}
			\term{Nervenzellen, die Informationsfluss zu ZNS verbinden}
			\definition{Zisterne}
			\definition{Anaphase}
			\definition{Maximum, Minimum}
			\definition{Glucose}
			\definition{PCR}
			\definition{X-Chrom.geb. Kreuzung Uniformität}
			\definition{weisse Substanz: Rückenmark Fkt}
			\definition{Kasper-Hauser-Experiment}
			\term{2-Chromatid-Chromosomen, die am Centromer zusammengehalten werden, werden auseinandergezogen}
			\term{Hohlröume in ER}
			\term{Traubenzucker}
			\term{Ober- / Unterhalb dieser Temp keine Aktivität}
			\term{dom auf X X fem}
			\term{Polymerase Chain Reaction. Massive vervielfältigung von DNA. Denaturierung bei 94 Grad, Primer bilden, Hitzebeständige Taq-Polyermase zur vervielfältigung einsetzen.}
			\term{Aufzucht von Individuuen unter Erfahrungsentzug; dient zur Feststellung, ob bestimmte Fähigkeiten angeboren sind oder erlernt werden, so können n! von Artgenossen. lernen}
			\term{efferente und afferente Bahnen laufen hierdurch}
			\definition{Dictyosom}
			\definition{Telophase}
			\definition{Gedeihkurve=Toleranzkurve}
			\definition{Chlorophyll}
			\definition{Telomer Ort}
			\definition{X-Chrom.geb. Kreuzung keine Uniformität}
			\definition{graue Substanz: Rückenmark Fkt}
			\definition{Ethogramm Analyse}
			\term{Dekondensation der 1-Chromatid-Chromosomen, neue Kernhüllen}
			\term{Membranstapel, Netz, --> GOLGI-Apparat}
			\term{Farbstoff in Chloroplast}
			\term{In welchem Ausmass ein Lebewesen auf Umwelt reagiert}
			\term{dom auf X male}
			\term{Enden von Chromosomen}
			\term{Verhaltensbereich, Beschreibung, Dauer, Häufigkeit, Ultimate Ursache, Proximate Ursache, Physiologie des Verhaltens, Etho-Phylogenie}
			\term{Durch Spinalganglion werden von Sinneszellen geleitete Signale in gr Substanz weitergeleitet, Hintere Wurzel sendet motorische Signale zurück.}
			\definition{Golgi-Apparat}
			\definition{Interphase}
			\definition{Ökologische Potenz}
			\definition{Fotosynthese Umweltfaktoren}
			\definition{Telomere Fkt}
			\definition{Genkopplung}
			\definition{Alles oder nichts Prinzip}
			\definition{Fehlerquellen Ethogramm}
			\term{Vorbereitung auf Zellteilung}
			\term{Golgi-Vesikel bei Dictyosomen, Aufbau Zellwandmaterial, Sekretstoffe}
			\term{Licht, Temperatur (mit Optimum), CO2-Gehalt}
			\term{Gesamte Reaktionsfähigkeit eines Lebewesens auf Umwelt}
			\term{Gene werden im Laufe von Generationen gemeinsam vererbt; Verhalten sich nicht im Laufe der dritten mendelschen Regel}
			\term{Schutzkapsel, Schutz vor Enzymen, Stabiltät, Pufferzone bei Replikation: Lücke bei DNA-Polymerase I:Nicht möglich an 5' enden abzuschliessen}
			\term{Zu geringe Anzahl beobachtungen, Subjektive Akzentuierung, unädequater Biotop}
			\term{Wird ein Schwellenwert überschritten, so folgt die Reaktion in voller Stärke}
			\definition{Plastide}
			\definition{Mensch Anzahl Chromosomen}
			\definition{eurypotent, euryök}
			\definition{Lichtkompensationspunkt}
			\definition{Telomerase}
			\definition{Koppelungsgruppe}
			\definition{Ruhepotenzial}
			\definition{angeborene Lerndisposition}
			\term{46; 23 Chromosomenpaare}
			\term{Begrenzt; Entwickeln sich zu Leukoplast (Amyloplast, Proteinoplast, Elaioplast), Chloroplast;}
			\term{Lichtintensität, wo die Pflanze gleich viel CO2 wie O2 abgibt}
			\term{In der Hinsicht eines Faktors ist das Lebewesens Toleranzbereich sehr weit}
			\term{Auf einem Chromosomen liegende Gene; verhalten sich wie ein Gen (kinda)}
			\term{Enzym, Regeneration von Telomeren.}
			\term{zb Schlangenfurcht; Furcht nicht angeboren aber die Bereitschaft, schnell eine Phobie zu entwicklen auf Schlangen; manche Sachen lernen Menschen schneller}
			\term{Spannung zwischen Aussen und innen, wenn aussen 0mV, dann innen -70mV}
			\definition{Amyloplast, Proteinoplast, Elaioplast}
			\definition{Diploid}
			\definition{stenopotent, stenök}
			\definition{Absorptionsspektren Licht Fotosynthese}
			\definition{Telomere Bedeutung}
			\definition{Morgan-Einheiten}
			\definition{Diffusionspotenzial}
			\definition{bereichsspezifische Fähigkeiten}
			\term{2n; Zwei Chromosomensätze}
			\term{Speicherung von Stärke, Protein, Lipide}
			\term{Präferieren Blaues und Rotes Licht. --> Grüne Farbe absorbiert blaues und rotes Licht}
			\term{In der Hinsicht eines Faktors ist das Lebewesens Toleranzbereich sehr schmal}
			\term{Prozentsatz von Entkoppelungen zweier Allele.}
			\term{Telomer bei jeder Zellteilung verkürzt und weniger regeneriert durch telomerase: Biologische Alterung, Zeigt alterung der zellfunktionen.}
			\term{Zb menschen können sehr gut eine sprache lernen wenn sie noch kinder sind}
			\term{Überschuss an positiver Ladung auf einer Seite der Membran, diffundieren.}
			\definition{Chloroplast}
			\definition{Haploid}
			\definition{endotherm <--> ektotherme Lebenwesen}
			\definition{Blattfarbstoffe}
			\definition{molecular clock}
			\definition{Chiasma}
			\definition{Membranpotenzial}
			\definition{Reizfilterung Ebenen}
			\term{1n; Einfacher Chromosomensatz}
			\term{Chlorophyll und Carotin absorbieren Licht; Photosynthese in Thylakoidbereichen}
			\term{Chlorophyll, $\beta$-Carotin}
			\term{gleichwarme und wechselwarme Tiere}
			\term{Überkreuzungsstelle}
			\term{Länge der Telomere bestimmt Anzahl möglicher Teilung einer Zelle.}
			\term{periphere Filterung, zentrale Filterung}
			\term{Postive und Negative Ladungen sind von einer Membran getrennt}
			\definition{Stroma}
			\definition{Meiose}
			\definition{endotherm}
			\definition{Thylakoid}
			\definition{genetischer Code. wtf is it?}
			\definition{Genkartierung Prinzip}
			\definition{K+ Sickerkanäle}
			\definition{periphere Filterung}
			\term{Zwei hintereinander ablaufende Reifeteilungen}
			\term{Inneres Chloroplast}
			\term{Innenraum des Chloroplasten}
			\term{produzieren Wärme selber}
			\term{Wahrscheinlichkeit Crossing Over erhöht bei weiten Abständen}
			\term{Basenabfolge}
			\term{Filterung direkt beim Rezeptor (Sinnesorgan)}
			\term{Es können K+ Ionen frei diffundieren, doch Na+ Teilchen bleiben ausserhalb}
			\definition{(Stroma-) Thylakoid}
			\definition{Stadien Meiose}
			\definition{ektotherm}
			\definition{Grana}
			\definition{Codon}
			\definition{Dreipunktanalyse}
			\definition{Ionenkanäle}
			\definition{Schlüsselreiz}
			\term{Interphase 2n, Prophase I 2n, Metaphase I 2n, Anaphase I 2n, Telophase 2n, KEINE INTERPHASE, Prophase II n, Metaphase II n, Anaphase II n, Telophase II n}
			\term{flache Membranzisterne, enthalten Cholorphyll und Carotin --> Photosynthese.}
			\term{Gestapelte Thylakoide}
			\term{sind auf äussere Wärmequellen angewiesen}
			\term{Austauschwerte gegen relative Abstände auf dem Chromosom an"Thema 11"}
			\term{Basentriplet. Codieren immer eine Aminosäure}
			\term{ein spezifischer Reiz, der ein bestimmtes (instinktives) Verhalten auslöst.}
			\term{Zur Auferhaltung dass K+ Ionen frei diffundieren können, aber Na+ Teilchen nicht}
			\definition{Enzyme aus was bestehen sie?}
			\definition{Interphase 2n}
			\definition{Kältetod, Hitzetod}
			\definition{Stroma}
			\definition{RNA Funktion}
			\definition{Ontogenie}
			\definition{Natrium-Kalium-Pumpe Fkt}
			\definition{zentrale Filterung}
			\term{Bereiten sich auf Zellteilung vor}
			\term{Aus Eiweissen}
			\term{Zwischenbereiche zwischen den Grana}
			\term{Temperatur-Toleranzbereich unter oder überschritten}
			\term{Entwicklung des Individuums von Befruchtung bis Tod}
			\term{Transport- und Arbeitskopie.}
			\term{Nervensystem bewertet Situationen}
			\term{Na+ Ionen aus intrazellulärem Raum raushalten, zur Aufrechterhaltung des Ruhepotenzials}
			\definition{Cofaktoren}
			\definition{Prophase I 2n}
			\definition{Bergmannsche Regel}
			\definition{Fotosysteme im Chloroplasten}
			\definition{RNA-Transkrtion}
			\definition{Embryologie}
			\definition{Hyperpolarisation, Depolarisation}
			\definition{Taxi}
			\term{Zwei homologe Zwei-Chromatide-Chromosom ordnen sich homolog an}
			\term{Neben Eiweissen haben Enzyme noch andere chemische Bestandteile; niedermolekulare Teile, Metallionen; werden durch Vitamine aufgenommen}
			\term{I und II}
			\term{Grosse Tiere haben im Vergleich zu ihrem Volumen eine kleine Köperoberfläche, darum verlieren sie weniger Wärme. Bsp pinguine}
			\term{Lehre der Embryonalentwicklung}
			\term{DNA in mRNA}
			\term{Orientierungsverhalten eines Individuums im Raum}
			\term{Spannung ist unter, über des Ruhepotentials}
			\definition{Enzym: Katalase}
			\definition{Metaphase I 2n}
			\definition{Allensche Regel}
			\definition{LHC}
			\definition{RNA-Polyermase}
			\definition{Begattung}
			\definition{Schwellenpotenzial}
			\definition{Voraussetzungen für Instinktverhalten}
			\term{Homologen Chromosomenpaare ordnen sich an der Äquatorialplatte an}
			\term{Wasserstoffperoxid zerlegt (H2O2)}
			\term{Antennenkomplexe, Light Harvestic Complex; von dem wird energie weitergeleitet}
			\term{Kleine Tiere haben im Vergleich zu ihrem Volumen eine grosse Köperoberfläche, darum verlieren sie mehr Wärme. Bsp fuchs}
			\term{Vereinigung des männlichen und weiblichen Individuums.}
			\term{Kopiert DNA in RNA (mRNA). 5'-->3'.}
			\term{Appetenzverhalten}
			\term{Falls Schwellenpotenzial erreicht, gibt es ein Aktionspotenzial mit Überschuss}
			\definition{Apoenzym}
			\definition{Anaphaes I 2n}
			\definition{Kamele Nasenschleimhaut}
			\definition{P700, P680}
			\definition{Promoter}
			\definition{Basmung}
			\definition{Grund für AP mit Überschuss}
			\definition{AAM}
			\term{Teilen sich am Centromer}
			\term{Der Proteinanteil des Enzyms}
			\term{Lichtsystem I, absorbiert 700nm-Licht, Lichtsystem II, absorbiert 680nm-Licht}
			\term{Beim Einatmen kühlt sich die Luft durch verdunstung}
			\term{Eindringen des Spermiums in Eizelle}
			\term{Erkennt Anfangsbasensequenz bei RNA-Transkrition}
			\term{angeborener auslösender Mechanismus. Filtert den Schlüsselreiz; Nur bei Instinkten bzw angeborenen Verhalten}
			\term{Ab Schwellenpotenzial öffnen sich spannungsgesteuerte Natrium-Ionen Kanäle --> Natriumionen stürmen in Zelle herein. Nach 1ms wieder geschlossen}
			\definition{prostehtische Gruppen = Holoenzym}
			\definition{Telophase I 2n}
			\definition{Entstehung Vogelzug}
			\definition{Primärprozesse}
			\definition{RNA-Translation}
			\definition{Befruchtung}
			\definition{Refraktärzeit}
			\definition{Übersprungverhalten}
			\term{Neue Hüllen --> 1n, haploid}
			\term{Apoenzyme eng mit Cofaktoren verbunden.}
			\term{lichtabhängige Prozesse; fotochemischer Teil, Lichtenergie --> chemische Energie}
			\term{Durch Abnehmende Tageslänge}
			\term{Verschmelzen der Zellkerne der weibl. + männl. Gameten}
			\term{mRNA --> Protein.}
			\term{Instinktverhalten aus anderem Bereich kommt zustande}
			\term{Direkt nach Auslösen des AP kann das Neuron keine weiteren APs erstellen (nach 1ms)}
			\definition{ATP wo genau Energie gespeichert}
			\definition{Prophase II n}
			\definition{Fotoperiodik}
			\definition{Sekundärprozesse}
			\definition{tRNA}
			\definition{vegetative Fortpflanzung}
			\definition{Restore Ruhepotenzial nach AP m Überschuss}
			\definition{Umorientierte Handlung}
			\term{Chromosomen ordnen sich am der Äquatorialplatte an}
			\term{In chemischen Bindungen von zwei Phospatresten P; Energie wird freigesetzt durch abspalten von Phospatresten. }
			\term{lichtunabhängige Prozesse, biochemischer Teil, der RGT-Regel unterliegend}
			\term{Gesang wird durch Sonnenaufgang ausgelöst, auch bei Pflanze}
			\term{ungeschlechtlich}
			\term{transfer-RNA: Transport von Aminosäuren zu Ribosomen}
			\term{Normal, aber gewisse Bedingungen verhindern, dass der Reiz ausgeführt werden kann.}
			\term{Durch Na K Pumpe}
			\definition{Hydrolyse ATP --> ...}
			\definition{Metaphase II n}
			\definition{circadiane Rhytmik}
			\definition{Zusammenfassung Primärprozesse}
			\definition{Wo RNA-Translation}
			\definition{generative Fortpflanzung}
			\definition{3 Potenziale}
			\definition{Leerlaufhandlung}
			\term{Chromosomen werden geteilt}
			\term{ADP: Adenosindiphosphat}
			\term{Lichtenergie in chemische Energie, in Thylakoidmembran Fotosysteme I und II (P700 und P680); Abgabe ATP und NADPH + H+ and Sekundärprozess; Freisetzung O2}
			\term{Aktiviäten in 24h Takt}
			\term{Spermien}
			\term{Ribosomen.}
			\term{Antrieb so stark, dass Instiktverhalten spontan ausgeführt wird.}
			\term{Ruhepotenzial, Aktionspotenzial, Nachpotenzial}
			\definition{Cosubstrat}
			\definition{Anaphase II n}
			\definition{Langtagpflanzen / Kurztagpflanzen}
			\definition{Zusammenfassung Sekundärprozesse}
			\definition{Anticodon}
			\definition{4 Stufen Embryonalentwicklung}
			\definition{Marklose Fasern Informationsübertragung}
			\definition{Intentionsbewegung}
			\term{Bilden neue Hüllen}
			\term{Coenzyme gehen jedes Mal verändert aus Reaktion hervor, verhält sich also wie ein Substrat. Also Cosubtrat. Bsp: ATP ist ein Cosubstrat}
			\term{Reduktion von Kohlenstoffdioxid CO2 zu Kohlenhydraten (CH2)n (=Zucker); Verbrauch ATP und NADPH + H+, Abgabe ADP und NADP+ an Primärprozess}
			\term{Blühen wenn Tag länger / kürzer ist}
			\term{Furchung}
			\term{Für eine Art t-RNA spezifisches Codon (Triplet).}
			\term{Eine nicht zu Ende geführtes Instinktverhalten}
			\term{Ausgleichsströme}
			\definition{ATP --> ADP Enzym}
			\definition{Telophase II n}
			\definition{Phytochromsystem}
			\definition{Motoren der Fotosynthese}
			\definition{Initiation RNA-Translation}
			\definition{Furchung}
			\definition{Markhaltige Fasern Informationsübertragung}
			\definition{Ritualisierung}
			\term{Fertig}
			\term{ATP-Synthase"Thema 02"}
			\term{Anregung von P680 und P700 (FS II + I).}
			\term{Lichtempfindliches System der Pflanzen}
			\term{Zellteilung}
			\term{t-RNA knüpft bei Start-Codon an, Ort P.}
			\term{Signalübermittlung bei der Kommunikation}
			\term{Ranvierrsche Schnürring zu Schnürring}
			\definition{Dichteanomalie des Wassers}
			\definition{Meiose Spermien <--> Ei Mutterzelle}
			\definition{Horizontal Zonierung See}
			\definition{Cytochrom-bf-Komplex Fkt}
			\definition{Elongation RNA-Translation}
			\definition{Blastocyste}
			\definition{Synapse Informationsübertragung}
			\definition{Prägung Merkmale}
			\term{Spermienmutterzelle --> 4 Spermien; Eimutterzelle --> 1 Eizelle + 3 Pollkörperchen}
			\term{4 Grad Celsius am Dichtestens}
			\term{Redoxsystem, das 1) Elektronen von P680 zu P700 transportiert; 2) Durch passieren der angeregten Elektronen ATP-Synthase ermöglicht.}
			\term{Benthal (=Litoral+Profundal)}
			\term{So nennt man eine Zelle die schon einen Hohlruam hat}
			\term{t-RNA knüpft bei Ort A an, unter Energieverbruach werden die beiden Aminosäuren verknüpft --> Dipeptid, vorrücken des Ribosoms.}
			\term{In einer sensiblen Phase entstanden, schnell, irreversibel}
			\term{Durch chem. Transmitterstoffe, AP nicht weitergeleitet}
			\definition{Dreischichter Aufbau bei allen Biomembranen}
			\definition{interchromosonale Rekombination}
			\definition{Litoral}
			\definition{Redoxsystem}
			\definition{Ribosom: Ort P}
			\definition{Gastrulation}
			\definition{exzitatorische Synapse}
			\definition{Lernen Arten}
			\term{Bei der Metaphase ordnen sich die Chromosomenpaare an, 0.5 Chance to get to the winning sperm cell}
			\term{--> Elementarmembran}
			\term{Protein, das ständig zwischen oxidierten und reduziertem Zustand wechselt.}
			\term{Uferzone; Grossseggenried + Von Wasser bedeckte Fläche; Es wachsen Pflanzen}
			\term{Entstehung von Ektoderm}
			\term{Verknüpfungsort}
			\term{Klassische Konditionierung, operante Konditionierung, Prägung, Habituation, Spiel, Neugier, kognitives Lernen}
			\term{Erregende Neurotransmitter}
			\definition{Selektiv permeabel}
			\definition{intrachromosonale Rekombination}
			\definition{Profundal}
			\definition{Fotophosphorylierung}
			\definition{Ribosom: Ort A}
			\definition{Neurulation}
			\definition{inhibitorische Synapse}
			\definition{klassische Konditionierung}
			\term{Überkreuzungen von benachbarten Chromosomenstränge, Crossing Over}
			\term{gleich semipermeabel, lässt bestimmte Stoffe durch oder hält diese zurück.}
			\term{Energie der Protonen, die durch Cytochrom-bf-Komplex von P680 zu P700 transportiert wird, wird zur ATP-Synthase verwendet.}
			\term{Tiefbodenzone; Keine Pflanzen wachsen dort}
			\term{Auch Morphogenese}
			\term{Erkennungsort}
			\term{Bedingte Reaktionen, Assoziation ein neuer bedingter Reflex zu einem unbedingten Reflex hinzugefügt werden kann}
			\term{Hemmede Neurotransmitter}
			\definition{Material Biomembrane}
			\definition{ungeschlechtliche Forpflanzung}
			\definition{Benthal}
			\definition{Protonengradient}
			\definition{Termination RNA-Translation}
			\definition{Organogenese}
			\definition{Transmitter}
			\definition{Generalisierung}
			\term{Nachkommen stimmen mit Eltern überein"Thema 04"}
			\term{Lipide und Proteine}
			\term{ungleiches Protonengradient: Innerhalb und ausserhalb der Membran}
			\term{Litoral + Profundal}
			\term{eindringen des Spermiums in Eizelle}
			\term{Stopcodon.}
			\term{Ähnliche Reize werden als einer aufgefasst}
			\term{Chemische Botenstoffe in synaptsichen Spalt}
			\definition{Aufbau Lipidmolekül}
			\definition{Erythrozyten}
			\definition{Vertikal Zonierung See}
			\definition{Fotolyse Fkt}
			\definition{Sichelzellenanämie}
			\definition{Vorgang Schwangerschaft}
			\definition{Gehirn: Schützende Schichten}
			\definition{Löschung}
			\term{Rote Blutkörperchen}
			\term{Lipophiler und Hydrophiler Teil}
			\term{Zerlegung des Wassers: 2 H20 --> O2 + 4 H+; Wird durch angeregtes P680 gespalten.}
			\term{Pelagial, tropogene Zone, Kompensationsebene, Zehrschicht}
			\term{ Folliker platzt}
			\term{Erkrankheit, Erythrozyten sind sichelförmig, Betroffene können nicht mehr gut Sauerstoff aufnehmen, Gefässe verstopfen, Milz vergrössert und platzt., Punktmutation}
			\term{Falls der unbedingte Reiz ausfällt, wird die Assoziation abgeschwächt}
			\term{[Tiefe] Weiche Hirnhaut > Spinngewebshaut > harte Hirnhaut > Schädel}
			\definition{Plasmalemma}
			\definition{Rote Farbe bei Erythrozyten}
			\definition{Pelagial}
			\definition{Ferredoxin}
			\definition{Introns, Exons}
			\definition{Folliker}
			\definition{Grosshirn Fkt}
			\definition{operantes Verhalten}
			\term{Hämoglobin}
			\term{Membran, die das Zellplasma begrenzt. = Zellmembran}
			\term{Eisenhaltiges Enzym; nimmt Elektronen von P700 auf, leitet es weiter an die NADP-Reduktase}
			\term{Zone des freien Wassers}
			\term{hat eine Eizelle in sich un platzt.}
			\term{Uncodierte Abschnitte der DNA: Intron, Codierte Abschnitte der DNA: Exons.}
			\term{spotan auftretendes Verhalten}
			\term{Gedächtnis, Sinnesleistung, Befehle an Muskeln}
			\definition{Glykokalyx}
			\definition{Funktion Hämoglobin}
			\definition{tropholytische Zone}
			\definition{NADP-Reduktase fkt}
			\definition{prä-mRNA}
			\definition{Klonen}
			\definition{graue Substanz: Gehirn Fkt}
			\definition{operante Konditionierung}
			\term{Sauerstoff binden, an Häm-Gruppe: Zentrum Eisenion}
			\term{Extrazelluläre Schicht des Plasmalemma, die Glykolipide und Glykoproteine enthält.}
			\term{Führt H+ und elektronen wieder zusammen: Cosubstart NADP+ nimmt elektronen und Protonen auf, entsteht NADPH + H+}
			\term{Zehrschicht}
			\term{Erzeugen identischer Nachkommen}
			\term{Trankskribierte RNA mit Intron included.}
			\term{Bedingte Aktionen, tier lernt, dass aktionen erfolg oder misserfolg haben}
			\term{Oberfläche, Neuronen angebracht}
			\definition{Plasmolyse}
			\definition{Leukozyten}
			\definition{trophogene Zone}
			\definition{Ergebnisse Primärprozess}
			\definition{Splicing}
			\definition{Klonen Typen}
			\definition{weisse Substanz: Gehirn Fkt}
			\definition{Lernen durch Prägung}
			\term{Weisse Blutkörperchen}
			\term{Zellen schrumpfen; das Cytoplasma wird rausgesogen.}
			\term{ATP und NADPH + H+ --> Weiter an sekundärprozess; O2 Abgabe. :)}
			\term{Nährschicht}
			\term{Horizontales Klonen}
			\term{Introns werden herausgeschnitten}
			\term{Prägung}
			\term{Inneres, Axone}
			\definition{Deplasmolyse}
			\definition{Thrombocyten}
			\definition{Bedeutung Nährschicht = trophogene Schicht}
			\definition{Wo Sekundärprozess}
			\definition{Epigenetische Regulation}
			\definition{Horizontales Klonen}
			\definition{Balken}
			\definition{Habituation}
			\term{Blutplättchen, Blutgerinnung}
			\term{Zellen wachsen, die Flüssigkeit wird in die Zellen reingedrückt.}
			\term{Stroma und Chloroplast}
			\term{Produziert Tagsüber Sauerstoff und Biomasse}
			\term{Klassisches Klonen. Zellteilung bis 8-Zell-Stadium}
			\term{Zb Bienen. Je nach den Verhältnissen kann sich der Genotyp ändern!}
			\term{Gewöhnung}
			\term{Verbindet die Hemisphären}
			\definition{Turgor}
			\definition{Blutplasma}
			\definition{Bedeutung Kompensationsschicht}
			\definition{Fixierung von CO2}
			\definition{Genomische Prägung}
			\definition{Vertikales Klonen}
			\definition{limbisches System}
			\definition{kognitives Lernen}
			\term{Lösung Proteine, Kohlenhydrate, Mineralien}
			\term{Druck gegen Zellwände}
			\term{Bindung CO2 and C5-Körper}
			\term{Sauerstoff- und Biomasseverbrauch heben sich gegenseitig auf}
			\term{Nukleus-Transfer in eine normale Eizelle}
			\term{In eine männlichen Embryo werden männliche Methylierungsmuster erzeugt, in einem weblichen webliche. (?) Auch umzu Erfahrungen weiterzugeben!}
			\term{Lernen durch Einsicht, Zusammenhänge begreifen...}
			\term{Bewusstsein, Emotionen}
			\definition{Brownsche Bewegung}
			\definition{Bildung der Blutzellen}
			\definition{Bedeutung Zehrschicht}
			\definition{Reduktion}
			\definition{differenzielle Genaktivität}
			\definition{Therapeutisches Klonen}
			\definition{Zwischenhirn: Welche Teile}
			\definition{Bedeutung Agression}
			\term{Knochenmark}
			\term{Temperaturabhängige Eigenbewegung der Teilchen}
			\term{Durch ATP-->ADP und NADPH + H+ --> NADP+ 6 CO2-Moeküle für Synthetisierung von Glucose (C6-Körper)}
			\term{Verbrauch>Produktion}
			\term{Klonen embryonaler Stammzellen für die Produktion von gesunden Zellen}
			\term{Bestimmte Gene werden zu festgelegten Zeitpunkten an- oder abgeschaltet}
			\term{rivalisieren um Ressourcen, selektionswert}
			\term{Epithalamus, Epiphyse, Hypothalamus, Hypophyse}
			\definition{Brownsche Bewegung Bedeutung}
			\definition{Erythrozyten Energiegewinnung}
			\definition{See Temperaturschichten}
			\definition{Rückbildung des Akkzeptors}
			\definition{Puffmuster}
			\definition{Reproduktives Klonen}
			\definition{Zwischenhirn Fkt}
			\definition{Kommentkampf}
			\term{Glykolyse, nicht Mitochrondrien ATP}
			\term{Diese Eigenbewegung bewirkt die Eigenverteilung der Stoffe bis zum Konzentrationsausgleich --> Diffusion}
			\term{Durch ATP-->ADP und NADPH + H+ --> NADP+ die instabilen Moleküle resynthesiert zu Akzeptoren (C5-Körper)}
			\term{Epilimnion --> Sprungschicht --> Hypomilion}
			\term{Herstellung eines genetischen identischen Embryos und Übertragen in fremde Mutter}
			\term{Riesenchromosomen: Aufgepufft = hohe transkriptionsrate}
			\term{wie Rituale, Kämpfe mit festgelegten Regeln}
			\term{Umschaltstelle Nervenbahnen, Koordination wie Schlafen, Atmung, Kreislauf, Sexualität}
			\definition{Diffusion}
			\definition{Leukozyten Arten}
			\definition{Stagnationszustände Jahreszeiten}
			\definition{Ergebnisse Sekundärprozess}
			\definition{Promotor-Regien bei RNA-Transkriptase}
			\definition{Dolly welche Art von Klonen}
			\definition{Mittelhirn Fkt}
			\definition{Beschädigungskämpfe}
			\term{Granulocyte, Makrophage, Lymphozyte}
			\term{Diese Eigenbewegung bewirkt die Eigenverteilung der Stoffe bis zum Konzentrationsausgleich }
			\term{Glucose, CO2 Synthetisiert.}
			\term{Sommerstagnation: Stabile Schichtung; Frühjahr+Herbst: Vollzirkulation; Winter: Winterstagnation}
			\term{Horizontales Klonen.}
			\term{dort setzt die RNA-Transkriptase an.}
			\term{DEATH}
			\term{Informationen aus Sinnen}
			\definition{Permeation}
			\definition{Leukämie}
			\definition{Drei Gruppen in Nahrungsketten}
			\definition{Weiterverarbeitung:}
			\definition{Enhancer}
			\definition{Stammzellen}
			\definition{Kleinhirn Fkt}
			\definition{Drei Theorien Agressionsverhalten}
			\term{Vermehrte Bildung von Leukocyten}
			\term{Diffusion durch Grenzschichten}
			\term{Glucose --> Stärke... Fette, Proteine, Farbstoffe, Giftstoffe, Harze}
			\term{Produzenten, Konsumenten, Destruenten}
			\term{Nicht spezialisierte Zellen}
			\term{Sequenzen, die die Aktivität von Promotoren erhöht."Thema 09}
			\term{Mensch ist agressiv, Mensch wird agressiv, Mensch ist+wird agressiv}
			\term{Bewegungsabläufe}
			\definition{Osmose}
			\definition{Erythropoietin}
			\definition{Produzenten}
			\definition{Abbau Stärke}
			\definition{DNA in fremde Zelle Vorgänge}
			\definition{Totipotente Stammzellen}
			\definition{Nachhirn}
			\definition{Mensch ist agressiv}
			\term{Hormon, vermehrte Erythrocytenbildung}
			\term{Permeation durch selektiv permeable Membran}
			\term{--> Disaccharose}
			\term{Pflanzen, die durch Fotosynthese Biostoffe produzieren}
			\term{Können sich zu einem eigenständigen Individuum entwickeln. Bsp 8-Zellstadium}
			\term{Isoliert, durch RE zerschnitten, Vektor-DNA durch RE zerschnitten, Verbinden durch DNA-Ligase, einschleusen}
			\term{Freud, Lorenz}
			\term{auch verlängertes Mark, Reflexe, Grundrhytmus für ein und ausatmen}
			\definition{osmotischer Druck}
			\definition{Sauerstofftransport}
			\definition{Konsumenten}
			\definition{Zellatmung}
			\definition{RE Funktion}
			\definition{Pluripotente Stammzellen}
			\definition{Plastizität}
			\definition{Mensch wird agressiv}
			\term{Lungenbläschen binden an Häm-Gruppe (<-- Hämoglobin (<-- Erythrocyt)); In Gewebe Diffusion, gibt Sauerstoff dem Myoglobin ab}
			\term{Wassermoleküle diffundeieren Richtung Salzlösung, um einen Konzentrationsausgleich zu erlangen}
			\term{aerober Abbau, ADP-->ATP}
			\term{Fleischfresser, Pflanzenfresser, Parasiten}
			\term{Können sich zu Zellen der drei Keimblätter entwickeln. Bsp Blastcyste}
			\term{Zerschneiden DNA an bestimmten Stellen}
			\term{Frustrations-Agressionstheorie, Lerntheorie}
			\term{Bestimmte Aufgaben einer Gehirnregion können durch eine andere übernommen werden}
			\definition{osmotischer Wert}
			\definition{Myoglobin}
			\definition{Destruenten}
			\definition{Betriebsstoffwechsel}
			\definition{klebrige Ende}
			\definition{Multipotenten Stammzellen}
			\definition{sensorische Felder}
			\definition{Frustrations-Agressionstheorie}
			\term{Farbstoff in Geweben, der hohes Bindungsvermögen für Sauerstoffmoleküle besitzt}
			\term{Konzentration an gelösten Stoffen innerhalb der Zelle}
			\term{Fette, Proteine --> Glucose --> (ADP --> ATP) --> ENERGY."Thema 08}
			\term{Detritusfresser+Mineralisieren}
			\term{Können sich zu Zellen einer bestimmten Linie entwickeln. Bsp Adulte Stammzellen}
			\term{DNA, die mit gleichen Enzymgeschnitten worden sind sind komplementär und können verkettet werden.}
			\term{Frustration ist die Störung einer zielgerichteten Aktivität --> Frustrator wird angegriffen}
			\term{Gehirnregionen, die Informationen aufnehmen, die von Sinnesorganen kommen}
			\definition{isotonisch, hypertonisch, hypotonisch}
			\definition{Kohlenstoff Atmung wie}
			\definition{Destritusfresser}
			\definition{Phänotyp}
			\definition{physikalische DNA-Übertragungsmethoden}
			\definition{Adulte Stammzellen woher nehmen}
			\definition{motorische Felder}
			\definition{Lerntheorie}
			\term{Kohlenstoffmoleküle diffundieren in Blutkapillare}
			\term{Konzentration ausserhalb der Zelle [entspricht / ist höher / ist tiefer] als in der Zelle.}
			\term{äusseres Erscheinungsbild}
			\term{Zersetzen Tierleichen und tote Pflanzen}
			\term{Nabelschnurblut und Knochenmark (Übermässige Prod. --> in blut)}
			\term{Elektroportation, Mikroinjektion, Partikelpistole, Liposome}
			\term{Agression führt zur Erreichung eines bestimmten Ziels}
			\term{Gehirnregionen, die Befehle an Muskeln weitergeben}
			\definition{einfache Diffusion, passiver Transport}
			\definition{Blutstillung Vorgang}
			\definition{Mineralisierer}
			\definition{Mitose}
			\definition{biologische DNA-Übertragungsmethoden}
			\definition{Direkte Rückprogrammierung}
			\definition{Assoziationsfelder}
			\definition{Mensch ist+wird agressiv}
			\term{Thrombocyten geben Protein ab: Plättchenfaktor, das weitere Thrombocyten andocken lässt.}
			\term{Wassermoleküle diffundieren durch Lipid-Doppelschicht-Membran}
			\term{Zellkernteilung mit Ergbenis diploide Zellen}
			\term{Zersetzen Biomaterial zu Mineralstoffen und Wasser}
			\term{Einschleusen in Hautzelle von Genen}
			\term{Viren + Agrobacterium}
			\term{Genetisch-Soziales Modell}
			\term{Verknüpfen Informationen}
			\definition{5 Transportmöglichkeiten durch Zellmembran}
			\definition{Blutgerinnung Vorgang}
			\definition{Kohlenstoffkreislauf Abgeber, Aufnehmer}
			\definition{Meiose}
			\definition{Elektroportation Übertragungsmethode}
			\definition{iPS-Zellen}
			\definition{Wernicke-, Broca-Bereich}
			\definition{Genetisch-Soz Modell 5 Bereiche}
			\term{Gerinnungsfaktoren --> Enzym Thrombin --> Fibrinogen --> Fibrin: Erstellt ein Netz aus Fibrin}
			\term{einfache Diffusion, tunnelvermittelte Diffusion, carriervermittelte Diffusion, aktiver Transport}
			\term{Zellkerntielung mit Ergebnis haploide Zellen}
			\term{Abgeber: Atmung, Verbrennung Wald, Industrie, Ozean Abgabe; Aufnahme: Pflanzen, Ozean (Fixierung)}
			\term{induzierte Pluripotente Stammzellen; Ergebnis von direkter Rückprogrammierung}
			\term{Elektrische Ladung --> Löcher in der Zellwand}
			\term{Gene, Physiologie(hormone), Gesamtorganismus (psychisch), sozial (rangordnung zb), ökologisch (=umgebung)}
			\term{Beim Bennen eines Gegenstandes: Wernicke: Wortfindung, Broca: Aussprache der Wörter.}
			\definition{Tunnel-Bildende Proteine}
			\definition{Thrombose}
			\definition{N-Kreislauf Seen Zufuhr}
			\definition{Phän}
			\definition{Mikroinjektion Übertragungsmethode}
			\definition{Embryonale Stammzellen Klonverfahren}
			\definition{Endrophine}
			\definition{Genetisch-Soz Modell 3 historische Dim}
			\term{schmerzhafte Blutstauung}
			\term{gleich erleichterte Diffusion, Innenwände Bildende Proteine, die Moleküle und gelöste Ionen bestimmter Grösse passieren lassen}
			\term{Merkmal, physische Eigenschaft eines Lebewesens..}
			\term{Fixierung Tiere und totes Laub}
			\term{Unspezifische geklonte Stammzellen aus Eizellen durch ivF}
			\term{Zellkern mit feinem Rohr durchstochen und rein injiiziert}
			\term{stammesgeschichtlich (rassen), kulturgeschtlich (rituale), ontogonetisch (persönliche erfahrungen)}
			\term{Lindern Schmerzempfindung durch Ausschüttung von hemmenden Neurotransmittern}
			\definition{Carrier}
			\definition{Thrombus}
			\definition{Stickstoff}
			\definition{Genotyp}
			\definition{Partikelpistole Übertragungsmethode}
			\definition{Kontra therapeutisches Klonen}
			\definition{Cortex}
			\definition{aktiver Transport}
			\term{Blutgerinsel}
			\term{gleich erleichterte Diffusion, Transportproteine binden kurzfristig den zu Transportierenden Stoff, spezifische Bindungsstellen}
			\term{Gesamt der in den Genen lokalisierten Erbinformationen}
			\term{N}
			\term{Embryonen werden hergestellt}
			\term{Goldpartikel + DNA in Zelle schiessen}
			\term{Carrier, die entgegen dem Konzentrationsgefälle Substrate transportieren können. Bsp Natrium-Kalium-Pumpe}
			\term{Rindenfeld des Gehirns}
			\definition{Embolie}
			\definition{Ammonium}
			\definition{Genom}
			\definition{Liposome Übertragungsmethode}
			\definition{Nutzen Stammzellen}
			\definition{Asymmetrie Hemisphären}
			\definition{primärer aktiver Transport}
			\definition{offener Blutkreislauf}
			\term{NH4}
			\term{Schliessen einer Blutarterie}
			\term{Lipide verschmelzen mit Zellschicht}
			\term{Gesamtheit der DNA einer Zelle}
			\term{Hemisphären sind für die Gegenüberliegende Körperhälfte zuständig}
			\term{ Prokontra}
			\term{Blut strömt frei zwischen den Räumen der Organe und versorgt so die Organe}
			\term{Binden und Freisetzen des Substrate}
			\definition{Nitrat}
			\definition{Chromatin}
			\definition{Viren Übertragungsmethode}
			\definition{Präntale Diagnostik: invasiv/n. invasiv}
			\definition{linke Hemisphäre Fkt}
			\definition{}
			\definition{}
			\definition{}
			\term{Komplex aus DNA und Proteinen; vgl Chromatinfaser}
			\term{NO2, NO3}
			\term{Invasiv: Fötale Zellen}
			\term{Virus schleust DNA ein}
			\term{}
			\term{Umwelt gerichtet, aktiv}
			\term{}
			\term{}



\end{document}